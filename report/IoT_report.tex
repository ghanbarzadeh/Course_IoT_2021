% Usefull codes

%\begin{figure}T_
%	\centering 
%	\includegraphics[width=0.5\columnwidth]{swallow.jpg}
%	\caption{Closed-Loop system with PID controller} 
%\label{fig:1} 
%\end{figure}
%
%\begin{enumerate}
%	\item Step input tracking
%	\item Changing reference tracking (Servo)
%	\item Disturbance rejection
%\end{enumerate}
%
%\begin{enumerate}[(\itshape a\normalfont)] % Sub-questions styled as italic letters
%	\item Suppose ``chuck" implies throwing.
%	\item Suppose ``chuck" implies vomiting.
%\end{enumerate}
%
%\begin{center}{\begin{minipage}{0.85\linewidth}
%\begin{lstlisting}[
%	caption={First-degree with time delay approximation},
%	captionpos=t,label={code:1}]
%system = tf(1,[1,9,23,15]);
%[y,t] = step(system);
%
%time_step = t(2)-t(1);
%u = ones(size(y));
%data = iddata(y,u,time_step);
%system_model = procest(data,'P1D');
%\end{lstlisting}
%\end{minipage}}\end{center}
%
%\begin{equation}
%	P(A|B) = \frac{P(B|A)P(A)}{P(B)}
%\label{eq:bayes}
%\end{equation}
%
%	\lstinputlisting[
%		caption=Luftballons Perl Script, % Caption above the listing
%		label=lst:luftballons, % Label for referencing this listing
%		language=Perl, % Use Perl functions/syntax highlighting
%		frame=single, % Frame around the code listing
%		showstringspaces=false, % Don't put marks in string spaces
%		numbers=left, % Line numbers on left
%		numberstyle=\tiny, % Line numbers styling
%	]{luftballons.pl}
%
%\begin{center}
%	\begin{tabular}{l l l}
%		\toprule
%		\textit{Per 50g} & Pork & Soy \\
%		\midrule
%		Energy & 760kJ & 538kJ\\
%		Protein & 7.0g & 9.3g\\
%		Carbohydrate & 0.0g & 4.9g\\
%		Fat & 16.8g & 9.1g\\
%		Sodium & 0.4g & 0.4g\\
%		Fibre & 0.0g & 1.4g\\
%		\bottomrule
%	\end{tabular}
%\end{center}
%
%\begin{table}[hbt]
%\caption{Genetic Algorithm Hyper-Parameters}
%\centering
%\begin{tabular}{l|c}
%\toprule
%Chromosome per generation & $100$ \\
%Encoding & Real \\
%Variables & $K_p,T_i,T_d$ \\
%Lower bounds & $[0, 0.001, 0]$ \\
%Upper bounds & $[200, 5, 0.8]$ \\
%Termination criteria & MaxGen=$300$ \\
%Selection technique & Stochastic uniform \\
%Crossover operation & Arithmatic \\
%Mutation probability & 0.01 \\
%\bottomrule
%\end{tabular}
%\label{tab:1}
%\end{table}
%
%\begin{figure}
%\centering
%\subfloat[Output y]{\includegraphics[width=0.75\columnwidth]{y3.eps}}\quad
%\subfloat[Control signal u]{\includegraphics[width=0.75\columnwidth]{u3.eps}}
%\caption[Simulation results of disturbance rejection]{Simulation results of disturbance rejection}
%\label{fig:1}
%\end{figure}

%----------------------------------------------------------------------------------------

\documentclass[
	12pt, % Default font size, values between 10pt-12pt are allowed
	%letterpaper, % Uncomment for US letter paper size
]{packages/fphw}

% packages

% template specific packages
\usepackage[utf8]{inputenc} % Required for inputting international characters
\usepackage[T1]{fontenc} % Output font encoding for international characters
%\usepackage{mathpazo} % Use the Palatino font

\usepackage{graphicx} % Required for including images

\usepackage{booktabs} % Required for better horizontal rules in tables

\usepackage{listings} % Required for insertion of code

\usepackage{enumerate} % To modify the enumerate environment


% my packages

\setlength\parindent{0pt}

%\usepackage[framed,numbered,autolinebreaks,useliterate]{packages/mcode}

\graphicspath{{Figures/}}

\usepackage{subfig} % Required for creating figures with multiple parts (subfigures)
\usepackage[table,xcdraw]{xcolor}
\usepackage{hyperref} % More descriptive referencing
\usepackage{xcolor}
\hypersetup{
    colorlinks,
    linkcolor={blue!50!black},
    citecolor={blue!50!black},
    urlcolor={blue!80!black}
}
\AtBeginDocument{\renewcommand{\ref}[1]{\autoref{#1}}}

\usepackage{float}

\usepackage{amsmath}

\definecolor{codegreen}{rgb}{0.25,0.5,0.5}
\definecolor{codegray}{rgb}{0.4,0.4,0.4}
\definecolor{codepurple}{rgb}{0.729,0.129,0.129}
\definecolor{backcolour}{rgb}{0.968,0.968,0.968}
\definecolor{rcolour}{rgb}{0.81,0.81,0.81}
\definecolor{impcolour}{rgb}{0,0.5,0}

\lstdefinestyle{mystyle}{
	language=Python,
	frame=single,
	framerule=1.5pt,
	rulecolor=\color{rcolour},
    backgroundcolor=\color{backcolour},   
    commentstyle=\color{codegreen},
    keywordstyle=\color{impcolour},
    numberstyle=\tiny\color{codegray},
    stringstyle=\color{codepurple},
    basicstyle=\ttfamily\footnotesize,
    breakatwhitespace=false,         
    breaklines=true,                 
    captionpos=t,                    
    keepspaces=true,                 
    numbers=left,                    
    numbersep=10pt,                  
    showspaces=false,                
    showstringspaces=false,
    showtabs=false,                  
    tabsize=4,
}

\lstset{style=mystyle}
\renewcommand{\lstlistingname}{In}

%----------------------------------------------------------------------------------------
%	ASSIGNMENT INFORMATION
%----------------------------------------------------------------------------------------

\title{Final Project Report} % Assignment title

\author{Armin Ghanbarzadeh, Yasamin Borhani, Mohammad Hoseinzadeh} % Student name

\date{23 Jan, 2022} % Due date

\institute{K. N. Toosi University of Technology\\ Faculty of Mechanical Engineering - Mechatronics Group } % Institute or school name

\class{IoT} % Course or class name

\instructor{Dr. Najafi} % Professor or teacher in charge of the assignment

%----------------------------------------------------------------------------------------

\begin{document}

\maketitle 
The code can be found at \url{https://github.com/ghanbarzadeh/Course_IoT_2021}.
%----------------------------------------------------------------------------------------
\section*{Part I}

\subsection*{Loading Data}

The dataset contains recorded data from a gearbox working normally and the same gearbox when it has a broken gear tooth. The data is recorded with 4 sensors, each corresponding to the vibration of a point on the gearbox body.
For each class we have a total of 10 excel files. Each of these files has the vibration of the gears at a specific load. We have first merged all the data for each class and in \ref{t1} and \ref{t2}, some statistical properties of the dataset is shown.

\begin{table}[H]
\centering
\begin{tabular}{
>{\columncolor[HTML]{FFFFFF}}c 
>{\columncolor[HTML]{FFFFFF}}r 
>{\columncolor[HTML]{FFFFFF}}r 
>{\columncolor[HTML]{FFFFFF}}r 
>{\columncolor[HTML]{FFFFFF}}r 
>{\columncolor[HTML]{FFFFFF}}r 
>{\columncolor[HTML]{FFFFFF}}r }	
\multicolumn{1}{r}{\cellcolor[HTML]{FFFFFF}{\color[HTML]{212121} \textbf{}}} & {\color[HTML]{212121} \textbf{Sensor 1}} & {\color[HTML]{212121} \textbf{Sensor 2}} & {\color[HTML]{212121} \textbf{Sensor 3}} & {\color[HTML]{212121} \textbf{Sensor 4}} & {\color[HTML]{212121} \textbf{load}} & {\color[HTML]{212121} \textbf{failure}} \\ \cline{2-7} 
\multicolumn{1}{c|}{\cellcolor[HTML]{FFFFFF}{\color[HTML]{212121} \textbf{count}}} & {\color[HTML]{212121} 1.02e+06} & {\color[HTML]{212121} 1.02e+06} & {\color[HTML]{212121} 1.02e+06} & {\color[HTML]{212121} 1.02e+06} & {\color[HTML]{212121} 1.02e+06} & {\color[HTML]{212121} 1.02e+06} \\
\multicolumn{1}{c|}{\cellcolor[HTML]{FFFFFF}{\color[HTML]{212121} \textbf{mean}}} & {\color[HTML]{212121} 3.70e-03} & {\color[HTML]{212121} -1.20e-03} & {\color[HTML]{212121} 8.14e-03} & {\color[HTML]{212121} -1.12e-04} & {\color[HTML]{212121} 4.58e+01} & {\color[HTML]{212121} 0.00e+00} \\
\multicolumn{1}{c|}{\cellcolor[HTML]{FFFFFF}{\color[HTML]{212121} \textbf{std}}} & {\color[HTML]{212121} 7.38e+00} & {\color[HTML]{212121} 4.43e+00} & {\color[HTML]{212121} 4.11e+00} & {\color[HTML]{212121} 4.52e+00} & {\color[HTML]{212121} 2.83e+01} & {\color[HTML]{212121} 0.00e+00} \\
\multicolumn{1}{c|}{\cellcolor[HTML]{FFFFFF}{\color[HTML]{212121} \textbf{min}}} & {\color[HTML]{212121} -5.87e+01} & {\color[HTML]{212121} -3.29e+01} & {\color[HTML]{212121} -2.92e+01} & {\color[HTML]{212121} -3.13e+01} & {\color[HTML]{212121} 0.00e+00} & {\color[HTML]{212121} 0.00e+00} \\
\multicolumn{1}{c|}{\cellcolor[HTML]{FFFFFF}{\color[HTML]{212121} \textbf{25\%}}} & {\color[HTML]{212121} -3.92e+00} & {\color[HTML]{212121} -2.40e+00} & {\color[HTML]{212121} -2.20e+00} & {\color[HTML]{212121} -2.38e+00} & {\color[HTML]{212121} 2.00e+01} & {\color[HTML]{212121} 0.00e+00} \\
\multicolumn{1}{c|}{\cellcolor[HTML]{FFFFFF}{\color[HTML]{212121} \textbf{50\%}}} & {\color[HTML]{212121} -1.15e-01} & {\color[HTML]{212121} 6.16e-02} & {\color[HTML]{212121} 5.24e-02} & {\color[HTML]{212121} 1.31e-01} & {\color[HTML]{212121} 5.00e+01} & {\color[HTML]{212121} 0.00e+00} \\
\multicolumn{1}{c|}{\cellcolor[HTML]{FFFFFF}{\color[HTML]{212121} \textbf{75\%}}} & {\color[HTML]{212121} 3.79e+00} & {\color[HTML]{212121} 2.49e+00} & {\color[HTML]{212121} 2.29e+00} & {\color[HTML]{212121} 2.52e+00} & {\color[HTML]{212121} 7.00e+01} & {\color[HTML]{212121} 0.00e+00} \\
\multicolumn{1}{c|}{\cellcolor[HTML]{FFFFFF}{\color[HTML]{212121} \textbf{max}}} & {\color[HTML]{212121} 5.67e+01} & {\color[HTML]{212121} 3.09e+01} & {\color[HTML]{212121} 2.51e+01} & {\color[HTML]{212121} 3.73e+01} & {\color[HTML]{212121} 9.00e+01} & {\color[HTML]{212121} 0.00e+00}
\end{tabular}
\caption{Healthy gearbox data statistics}
\label{t1}
\end{table}

\begin{table}[H]
\centering
\begin{tabular}{
>{\columncolor[HTML]{FFFFFF}}c 
>{\columncolor[HTML]{FFFFFF}}r 
>{\columncolor[HTML]{FFFFFF}}r 
>{\columncolor[HTML]{FFFFFF}}r 
>{\columncolor[HTML]{FFFFFF}}r 
>{\columncolor[HTML]{FFFFFF}}r 
>{\columncolor[HTML]{FFFFFF}}r }
\multicolumn{1}{r}{\cellcolor[HTML]{FFFFFF}{\color[HTML]{212121} \textbf{}}} & {\color[HTML]{212121} \textbf{Sensor 1}} & {\color[HTML]{212121} \textbf{Sensor 2}} & {\color[HTML]{212121} \textbf{Sensor 3}} & {\color[HTML]{212121} \textbf{Sensor 4}} & {\color[HTML]{212121} \textbf{load}} & {\color[HTML]{212121} \textbf{failure}} \\ \cline{2-7} 
\multicolumn{1}{c|}{\cellcolor[HTML]{FFFFFF}{\color[HTML]{212121} \textbf{count}}} & {\color[HTML]{212121} 1.01e+06} & {\color[HTML]{212121} 1.01e+06} & {\color[HTML]{212121} 1.01e+06} & {\color[HTML]{212121} 1.01e+06} & {\color[HTML]{212121} 1.01e+06} & {\color[HTML]{212121} 1.01e+06} \\
\multicolumn{1}{c|}{\cellcolor[HTML]{FFFFFF}{\color[HTML]{212121} \textbf{mean}}} & {\color[HTML]{212121} -1.04e-03} & {\color[HTML]{212121} 1.73e-03} & {\color[HTML]{212121} 7.31e-04} & {\color[HTML]{212121} 1.34e-03} & {\color[HTML]{212121} 4.55e+01} & {\color[HTML]{212121} 1.00e+00} \\
\multicolumn{1}{c|}{\cellcolor[HTML]{FFFFFF}{\color[HTML]{212121} \textbf{std}}} & {\color[HTML]{212121} 4.60e+00} & {\color[HTML]{212121} 4.39e+00} & {\color[HTML]{212121} 3.81e+00} & {\color[HTML]{212121} 4.41e+00} & {\color[HTML]{212121} 2.90e+01} & {\color[HTML]{212121} 0.00e+00} \\
\multicolumn{1}{c|}{\cellcolor[HTML]{FFFFFF}{\color[HTML]{212121} \textbf{min}}} & {\color[HTML]{212121} -2.53e+01} & {\color[HTML]{212121} -3.25e+01} & {\color[HTML]{212121} -2.59e+01} & {\color[HTML]{212121} -2.74e+01} & {\color[HTML]{212121} 0.00e+00} & {\color[HTML]{212121} 1.00e+00} \\
\multicolumn{1}{c|}{\cellcolor[HTML]{FFFFFF}{\color[HTML]{212121} \textbf{25\%}}} & {\color[HTML]{212121} -2.77e+00} & {\color[HTML]{212121} -2.47e+00} & {\color[HTML]{212121} -2.03e+00} & {\color[HTML]{212121} -2.37e+00} & {\color[HTML]{212121} 2.00e+01} & {\color[HTML]{212121} 1.00e+00} \\
\multicolumn{1}{c|}{\cellcolor[HTML]{FFFFFF}{\color[HTML]{212121} \textbf{50\%}}} & {\color[HTML]{212121} -5.43e-02} & {\color[HTML]{212121} 1.26e-01} & {\color[HTML]{212121} 4.50e-02} & {\color[HTML]{212121} 1.01e-01} & {\color[HTML]{212121} 5.00e+01} & {\color[HTML]{212121} 1.00e+00} \\
\multicolumn{1}{c|}{\cellcolor[HTML]{FFFFFF}{\color[HTML]{212121} \textbf{75\%}}} & {\color[HTML]{212121} 2.67e+00} & {\color[HTML]{212121} 2.68e+00} & {\color[HTML]{212121} 2.10e+00} & {\color[HTML]{212121} 2.44e+00} & {\color[HTML]{212121} 7.00e+01} & {\color[HTML]{212121} 1.00e+00} \\
\multicolumn{1}{c|}{\cellcolor[HTML]{FFFFFF}{\color[HTML]{212121} \textbf{max}}} & {\color[HTML]{212121} 2.64e+01} & {\color[HTML]{212121} 2.47e+01} & {\color[HTML]{212121} 2.69e+01} & {\color[HTML]{212121} 3.24e+01} & {\color[HTML]{212121} 9.00e+01} & {\color[HTML]{212121} 1.00e+00}
\end{tabular}
\caption{Broken tooth gearbox data statistics}
\label{t2}
\end{table}

\subsection*{Data inspection}

We can plot the data for healthy and broken gearbox to get a better understanding. 

\begin{figure}[H]
\centering
\subfloat{\includegraphics[width=0.80\columnwidth]{s1}}
\caption{Sensor 1 data in different time-windows for healthy and broken gear}
\end{figure}

\begin{figure}[H]
\centering
\subfloat{\includegraphics[width=0.80\columnwidth]{s2}}
\caption{Sensor 2 data in different time-windows for healthy and broken gear}
\end{figure}

\begin{figure}[H]
\centering
\subfloat{\includegraphics[width=0.80\columnwidth]{s3}}
\caption{Sensor 3 data in different time-windows for healthy and broken gear}
\end{figure}

\begin{figure}[H]
\centering
\subfloat{\includegraphics[width=0.80\columnwidth]{s4}}
\caption{Sensor 4 data in different time-windows for healthy and broken gear}
\end{figure}

\subsection*{Data preparation}

Using the code below, we can split the dataset into train-test. Note that the split ratio is 80\%-20\%. 

\begin{center}{\begin{minipage}{0.9\linewidth}
\begin{lstlisting}[language=Python, basicstyle=\fontsize{10}{10}\selectfont\ttfamily]
from sklearn.model_selection import train_test_split

# The last element contains the labels
labels = healthy_df['failure'].values

# The other data points are the electrocadriogram data
data = healthy_df[['a1', 'a2', 'a3', 'a4']].values

X_train, X_test, y_train, y_test = train_test_split(
    data, labels, test_size=0.2, random_state=21
)
\end{lstlisting}
\end{minipage}}\end{center}

We have also normalized the dataset to be between 0-1 using the formula below. 

\begin{equation}
X = \frac{X-X_{min}}{X_{max}-X_{min}}
\end{equation}

\begin{center}{\begin{minipage}{0.9\linewidth}
\begin{lstlisting}[language=Python, basicstyle=\fontsize{10}{10}\selectfont\ttfamily]
min_val = tf.reduce_min(X_train)
max_val = tf.reduce_max(X_train)

X_train = (X_train - min_val) / (max_val - min_val)
X_test = (X_test - min_val) / (max_val - min_val)

X_train = np.array(X_train)
X_test = np.array(X_test)
\end{lstlisting}
\end{minipage}}\end{center}

The dataset is a time series and we can use RNN or LSTM networks. For this matter, we take 200 consecutive data points and concatenate them to become one single data. So now we have a total of 4063 train data (time series data) and 1015 test data. 

\subsection*{Neural network models}


When training the network, we would only use the correct (healthy) data. This is because in anomaly detection, the auto-encoder network learns to reconstruct a certain time series signal. 

\subsubsection*{LSTM autoencoder model} The autoencoder network first reduces the input size of (200,4) to (200, 16) and then (4). After we repeat this vector to obtain a size of (200,4). This then becomes (200, 16) and finally (200,4) which is the input dimension. 

\begin{figure}[H]
\centering
\subfloat{\includegraphics[width=0.9\columnwidth]{LSTM}}
\caption{LSTM-autoencoder neural network architecture}
\end{figure}

The model code is shown below. 

\begin{center}{\begin{minipage}{0.9\linewidth}
\begin{lstlisting}[language=Python, basicstyle=\fontsize{10}{10}\selectfont\ttfamily]
inputs = Input(shape=(200, 4))
L1 = LSTM(16, activation='relu', return_sequences=True, 
            kernel_regularizer=regularizers.l2(0.00))(inputs)
L2 = LSTM(4, activation='relu', return_sequences=False)(L1)
L3 = RepeatVector(200)(L2)
L4 = LSTM(4, activation='relu', return_sequences=True)(L3)
L5 = LSTM(16, activation='relu', return_sequences=True)(L4)
output = TimeDistributed(Dense(4))(L5)    
model = Model(inputs=inputs, outputs=output)

model.compile(optimizer='adam', loss='mae')
model.summary()

nb_epochs = 150
batch_size = 1024
history = model.fit(data_train, data_train, epochs=nb_epochs, batch_size=batch_size, validation_split=0.05).history
\end{lstlisting}
\end{minipage}}\end{center}

\subsubsection*{CNN autoencoder model}

Another network that can be a suitable candidate for the problem is a 1D CNN-autoencoder network. 1D CNN can be used to analyze time series data. 

The CNN-autoencoder network has 3 1D-CNN layers to reduce the data dimension ($200\rightarrow128\rightarrow64\rightarrow32$), and 3 1D-CNN transpose layers to decode ($32\rightarrow64\rightarrow128\rightarrow200$). 

\begin{figure}[H]
\centering
\subfloat{\includegraphics[width=0.9\columnwidth]{CNN}}
\caption{CNN-autoencoder neural network architecture}
\end{figure}

The model code is shown below. 

\begin{center}{\begin{minipage}{0.9\linewidth}
\begin{lstlisting}[language=Python, basicstyle=\fontsize{10}{10}\selectfont\ttfamily]
inputs = Input(shape=(200, 4))
x = Conv1D(128,3, activation='relu', dilation_rate=2)(inputs)
x2 = Conv1D(64,3, activation='relu', dilation_rate=2)(x)
x3 = Conv1D(32,3, activation='relu', dilation_rate=2)(x2)

d3 = Conv1DTranspose(32 ,3,activation='relu', dilation_rate=2)(x3)
d4 = Conv1DTranspose(64,3,activation='relu', dilation_rate=2)(d3)
d5 = Conv1DTranspose(128,3,activation='relu', dilation_rate=2)(d4)

decoded = Conv1D(4,1,strides=1, activation='sigmoid')(d5)
model= Model(inputs, decoded)
model.compile(optimizer='adam', loss='mae')
model.summary()

nb_epochs = 150
batch_size = 1024
history = model.fit(data_train, data_train, epochs=nb_epochs, batch_size=batch_size, validation_split=0.05).history
\end{lstlisting}
\end{minipage}}\end{center}

Adam optimizer usually has good results in training all kinds of neural networks, so we have chosen that here. The loss function is also mean absolute error (MAE) since we have a regression problem.
The batch size and number of epochs were selected to be 1024 and 150 for both CNN and LSTM networks.

\subsection*{Train Results}

We can plot the loss vs epochs in both networks. 

\begin{figure}[H]
\centering
\subfloat{\includegraphics[width=0.8\columnwidth]{LSTMa}}
\caption{LSTM-autoencoder loss vs epochs}
\label{fig7}
\end{figure}

\begin{figure}[H]
\centering
\subfloat{\includegraphics[width=0.8\columnwidth]{CNNa}}
\caption{CNN-autoencoder loss vs epochs}
\label{fig8}
\end{figure}

As it can be seen in \ref{fig7} and \ref{fig8} the loss decrease smoother in LSTM compared to the CNN network but CNN can achieve lower loss overall (CNN val-loss is 0.0067 compared to LSTM val-loss of 0.038).

\subsection*{Histogram of healthy and broken tooth gearbox data}

Histogram diagrams of LSTM and CNN networks for both correct and anomaly data is shown in \ref{fig9} and \ref{fig10} respectively.

\begin{figure}[H]
\centering
\subfloat[Healthy gearbox]{\includegraphics[width=0.45\columnwidth]{LSTMc.png}}\quad
\subfloat[Anomaly]{\includegraphics[width=0.45\columnwidth]{LSTMan.png}}
\caption{histogram diagram for loss function of LSTM network}
\label{fig9}
\end{figure}

\begin{figure}[H]
\centering
\subfloat[Healthy gearbox]{\includegraphics[width=0.45\columnwidth]{CNNc.png}}\quad
\subfloat[Anomaly]{\includegraphics[width=0.45\columnwidth]{CNNan.png}}
\caption{histogram diagram for loss function of CNN network}
\label{fig10}
\end{figure}

\subsection*{Results analysis}

These observations can be made from the histograms

\begin{enumerate}
	\item The CNN network has a smaller final loss compared to LSTM 
	\item Number of correct examples were more than anomaly examples
	\item both networks have less total loss for correct examples than anomaly examples
\end{enumerate}

So both networks trained how to reconstruct the correct examples (data or signals) with lower error(loss). Since they cannot reconstruct the anomaly examples, they will have a higher output error. We must choose a threshold for error 
To summarize, when we give an input to the network, the network reconstructs it and we can calculate the reconstruction error. We need to define a threshold to classify examples with less error than that to be correct and errors higher than that to be anomaly. 

To calculate this threshold, we used formula shown.

\begin{equation}
Threshold = MEAN(losses)+STD(losses)
\end{equation}

The evaluation results of networks on the test data are listed in \ref{t3},

\begin{table}[H]
\centering
\begin{tabular}{
>{\columncolor[HTML]{FFFFFF}}l 
>{\columncolor[HTML]{FFFFFF}}c 
>{\columncolor[HTML]{FFFFFF}}c }
{\color[HTML]{212121} \textbf{}} & {\color[HTML]{212121} \textbf{CNN NETWORK}} & {\color[HTML]{212121} \textbf{LSTM NETWORK}} \\ \cline{2-3} 
\multicolumn{1}{l|}{\cellcolor[HTML]{FFFFFF}{\color[HTML]{212121} \textbf{accuracy}}} & {\color[HTML]{212121} 92.5\%} & {\color[HTML]{212121} 81.4\%} \\
\multicolumn{1}{l|}{\cellcolor[HTML]{FFFFFF}{\color[HTML]{212121} \textbf{precision}}} & {\color[HTML]{212121} 100\%} & {\color[HTML]{212121} 79\%} \\
\multicolumn{1}{l|}{\cellcolor[HTML]{FFFFFF}{\color[HTML]{212121} \textbf{Recall}}} & {\color[HTML]{212121} 85\%} & {\color[HTML]{212121} 85.5\%} \\
\multicolumn{1}{l|}{\cellcolor[HTML]{FFFFFF}{\color[HTML]{212121} \textbf{F1}}} & {\color[HTML]{212121} 92\%} & {\color[HTML]{212121} 82\%}
\end{tabular}
\caption{Evaluation of networks}
\label{t3}
\end{table}

As it can be seen in \ref{t3}, the CNN network has better performance compared to LSTM. Speed of training for LSTM networks on average was $1 \frac{s}{step}$ and for CNN network was $96 \frac{ms}{step}$.

\subsection*{Sending Email}

After training the model, we can use it to predict any signal data from a working gearbox. After prediction, if the reconstruction error was higher than the threshold calculated earlier, we can send an email to the clients indicating that there is a problem with the gearbox. 
For demonstration we used the email address arduinokntu@gmail.com as server and najafi@kntu.ac.ir and j.khorramdel96@gmail.com as clients.

The code for this part is shown below.

\begin{center}{\begin{minipage}{0.9\linewidth}
\begin{lstlisting}[language=Python, basicstyle=\fontsize{10}{10}\selectfont\ttfamily]
import smtplib

reconstructions = model.predict(data_train[0:1,:,:])
loss = tf.keras.losses.mae(reconstructions, data_train[0:1,:,:])
loss=np.mean(loss)

if (loss>threshold):
   server=smtplib.SMTP_SSL("smtp.gmail.com",465)
   server.login("arduinokntu@gmail.com","xxxxxxxxx")
   server.sendmail("arduinokntu@gmail.com", 
   ["najafi@kntu.ac.ir","j.khorramdel96@gmail.com"], 
   "this is an email from [hoseinzadeh,borhani,ghanbarzadeh] IOT group.")
\end{lstlisting}
\end{minipage}}\end{center}

\subsection*{Conclusion}

In this report we trained a CNN and a LSTM network for anomaly detection. CNN can achieve better results with higher speed of training in comparison with LSTM. At the end if network recognize an anomaly signal, an email is sent from server to clients.

\newpage
\section*{Part II}

%----------------------------------------------------------------------------------------
%------------------------------------------------

\end{document}
